\hypertarget{introduction}{%
\section{Introduction}\label{introduction}}


\hypertarget{about_cantina}{
\subsection{About Cantina}\label{about cantina}}
Cantina is a security services marketplace that connects top security
researchers and solutions with clients. Learn more at
\href{https://cantina.xyz}{cantina.xyz}


\hypertarget{disclaimer}{
\subsection{Disclaimer}\label{disclaimer}}
A competition provides a broad evaluation of the security posture of the code at a particular moment
based on the information available at the time of the review. While competitions endeavor to identify
and disclose all potential security issues, they cannot guarantee that every vulnerability will be detected
or that the code will be entirely secure against all possible attacks. The assessment is conducted based
on the specific commit and version of the code provided. Any subsequent modifications to the code may
introduce new vulnerabilities, therefore, any changes made to the code would require an additional security review. Please be advised that competitions are not a replacement for continuous security measures
such as penetration testing, vulnerability scanning, and regular code reviews.
% Unused: Coinbase   stablecoin liquidity



% SUBSECTION: Risk Assessment
\hypertarget{risk-assessment}{%
\subsection{Risk assessment}\label{risk-assessment}}


% Width of table lines to 0.5mm
\setlength{\arrayrulewidth}{0.3mm}
% Padding inside cell
\renewcommand{\arraystretch}{1.1}


% --> Severity Classiification table: <--
\begin{center}

\begin{tabular}{|l|l|l|l|}
  \hline \textbf{Severity level} & \textbf{Impact: High} & \textbf{Impact: Medium} & \textbf{Impact: Low} \\
  \hline \textbf{Likelihood: high} & Critical & High & Medium \\
  \hline \textbf{Likelihood: medium} & High & Medium & Low \\
  \hline \textbf{Likelihood: low} & Medium & Low & Low \\
  \hline
\end{tabular}
\end{center}


% SUB-SUB-SECTIONS
\hypertarget{severity}{%
\subsubsection{Severity Classification}\label{severity}}

The severity of security issues found during the security review is 
categorized based on the above matrix.
High severity findings represent the most critical issues that must be addressed immediately,
as they either have high impact and high likelihood of occurrence, or medium impact with high likelihood.

Medium severity findings represent issues that, while not immediately critical, still pose
significant risks and should be addressed promptly. These typically involve scenarios with
medium impact and medium likelihood, or high impact with low likelihood.

Low severity findings represent issues that, while not posing immediate threats, could
potentially cause problems in specific scenarios. These typically involve medium impact
with low likelihood, or low impact with medium likelihood.

Lastly, some findings might represent improvements that don't directly impact
security but could enhance the codebase's quality, readability, or efficiency (Gas and Informational findings).

